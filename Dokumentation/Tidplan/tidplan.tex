% !TEX encoding = UTF-8 Unicode
\documentclass[a4paper]{article}



% For mac only
\usepackage[T1]{fontenc}
\usepackage[utf8]{inputenc}
\usepackage[swedish]{babel}


% 

\usepackage{amsmath}
\usepackage{graphicx}
\usepackage[colorinlistoftodos]{todonotes}

\usepackage{listings}


\title{Projekt i Reglerteknik FRT090\\ Tidplan: Grupp B}

\author{Johan Kellerth Fredlund, Koshin Aliabase\\ Santiago Castro, Sadik Sulejmanovic}

\date{\today}

\begin{document}
\maketitle



\section{Översikt}

\textbf{Design and Control of a Lego Unicycle}

I detta projekt ska vi bygga en enhjuling i lego som kan hålla balansen i både mediala och den laterala riktningen. Vi använder ett tröghetshjul för att se till att inte enhjulingen faller åt sidan. Vi tänkte modellera två processer, en för regleringen av tröghetshjulet och en för regleringen av det andra hjulet. Processen av den mediala riktningen liknar en inverterad pendulum vilket vi tänker utgå ifrån. Processen av den laterala riktningen liknar också en inverterad pendulum förutom hjulet längst ut.

Vi tänker använda oss av ett "complementary filter" för att skatta vinklarna och PID regulatorer för att reglera styrsignalen till respektive process. I slutändan tänker vi kombinera båda regulatorer så att enhjulingen kan balansera i båda riktningar samtidigt.

Test av regulatordesignen görs i simulink samt med hjälp av nyquist och bodediagram. En ev3 mikrokontroller kodas i C för att styra processen. Modelleringen av processen görs med hjälp av lagrange mekanikekvationerna, och lineariseras därefter.


\section{Arbetsområden}

    \subsection*{Regulatordesign}
    
    \begin{itemize}
    \item [] Design av regulatorn för tröghetshjulet
    \item [] Design av regulatorn för enhjulingen
    \end{itemize}
    
    \subsection*{Mjukvarudesign}
    
    \begin{itemize}
    \item [] Implementera regulatorn i C
    \end{itemize}
    
    
\section{Arbetsbörda}

\begin{itemize}
\item[] Modellering av processen: 1 vecka

\item[] Bygge av enhjulingen: 1 vecka

\item[] Design av regulatorn: 3 veckor

\item[] Implementering av regulatorn: 3 veckor
\end{itemize}

\section{Material}



\subsection*{Sensorer}
\begin{enumerate}
\item [] Gyroscope x2
\item [] Accelerometer
\end{enumerate}

\subsection*{Mikrokontroller}
\begin{enumerate}
\item [] ev3
\end{enumerate}

\subsection*{Motorer}

\begin{enumerate}
\item [] Legomotorer x4
\end{enumerate}

\subsection*{Generellt}

\begin{enumerate}
\item [] Hjul
\item [] Legokomponenter för bygge
\item [] AA batterier x6
\end{enumerate}

\newpage

\section{Veckoplanering}

	\subsection*{Vecka 1}
    
    \begin{itemize}
    \item Introduktion, val av projekt samt möte med handledare.
	\end{itemize}

	\subsection*{Vecka 2}
	% MAIN BEGIN
	\begin{itemize}
    		\item Tidplanen lämnas in (Tisdag)
    		\item Bygge av enhjulingen
		\item Förstå ev3
	\begin{itemize}
		\item Få input och output att fungera med ev3
		\item Bestämma vilket program som skall användas med ev3 (leJOS etc.)
		\item Installera programmet på huvuddatorn och testa så det funkar med ev3
	\end{itemize}

		
    		\item Bygge av reaktionshjulet
    	
    	\begin{itemize}
    \item Simulering av Inertia wheel system i simulink
    		\item Få en cirkelskiva av handledaren
		\item Testning av cirkelskivan enligt modelleringen praktiskt
    	\end{itemize}
    \item Modellering av processen för tröghetshjulet
    		\begin{itemize}
			\item Bestäm differential ekvationen för processen
    			\item Bestäm parametrar.
		\end{itemize}        
	
	\item Få motorerna att fungera
	\begin{itemize}
		\item Få motorn att fungera med programmet
		\item Få motorerna i synk två och två enligt nuvarande design (20141111)
	\end{itemize}
	\item Test av C kodningen
	
	\item Skriva klart rapporten inför måndag
	
	\end{itemize}
	% MAIN END
	
	

	
    
    \subsection*{Vecka 3}
    \begin{itemize}
    \item Lämna in första rapporten (måndag).
	\item Bestäm krav för regulatorn
    \item Få sensorerna att fungera
    \begin{itemize}
    	\item Testa sensor gyroskop
	\item Testa sensor accelerometer
	\item Undersöka om vi behöver flera sensorer
    \end{itemize}
    \item Utföra vinkel skattning med gyroskop sensorn (enkelt experiment)
    \item Design av regulatorn för lateral balans.
    \item Design av regulatorn för enhjulingen. 
    \item Test av regulatorn via simulink.
    \item Börja på implementringskoden. Allmänt test så att komponenterna fungerar samtidigt.
    \item Test av lateralbalansen via den riktiga processen.
	\end{itemize}
    
    \subsection*{Vecka 4}
    \begin{itemize}
	\item Design av regulatorerna, ska bli klart.
    \item Test av designen.
    \item Implementera de två regulatorerna.
    \item Testa de två processerna för sig.
    \item Båda processerna ska kunna regleras bra, kombinera.
	\end{itemize}
    
    \subsection*{Vecka 5}
    \begin{itemize}
    \item Skicka in andra rapporten (Fredag)
    \item Slutlig design av den kombinerande regulatorn.
    \item Slutligt test av regulatorns design.
	\item Implementation av regulatorn.
	\item Börja skriv slutrapporten. 
	\end{itemize}
    
    \subsection*{Vecka 6}
    \begin{itemize}
	\item Slutlig implementation av regulatorn.
    \item Förbättra regulatordesignen vid behov.
    \item Skriv den slutliga rapporten.
    \item Lägg till tillägg i mån av tid.
	\end{itemize}
    
    \subsection*{Vecka 7}
    \begin{itemize}
	\item Test av processen.
    \item Testa hjulen oberoende och tillsammans
    \item Skriv klart slutlig rapport.
	\end{itemize}

9 Januari: Skicka in slutrapport


\end{document}
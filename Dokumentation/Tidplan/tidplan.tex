\documentclass[a4paper]{article}

\usepackage[english]{babel}
\usepackage[utf8]{inputenc}
\usepackage{amsmath}
\usepackage{graphicx}
\usepackage[colorinlistoftodos]{todonotes}

\title{Tidplan}

\author{Johan Kellerth Fredlund, Koshin Aliabase\\ Santiago Castro, Sadik Sulejmanovic}

\date{\today}

\begin{document}
\maketitle



\section{Veckoplanering}

	\subsection*{Vecka 1}
    
    \begin{itemize}
    \item Introduktion, val av projekt samt möte med handledare.
	\end{itemize}

	\subsection*{Vecka 2}
	% MAIN BEGIN
	\begin{itemize}
    		\item Tidplanen lämnas in (Tisdag)
    		\item Bygge av enhjulingen
		\item Förstå ev3
	\begin{itemize}
		\item Få input och output att fungera med ev3
		\item Bestämma vilket program som skall användas med ev3 (leJOS etc.)
		\item Installera programmet på huvuddatorn och testa så det funkar med ev3
	\end{itemize}

		
    		\item Bygge av reaktionshjulet
    	
    	\begin{itemize}
    \item Simulering av Inertia wheel system i simulink
    		\item Få en cirkelskiva av handledaren
		\item Testning av cirkelskivan enligt modelleringen praktiskt
    	\end{itemize}
    \item Modellering av processen
    		\begin{itemize}
			\item Bestäm differential ekvationen för processen
    			\item Bestäm parametrar.
		\end{itemize}        
	
	\item Få motorerna att fungera
	\begin{itemize}
		\item Få motorn att fungera med programmet
		\item Få motorerna i synk två och två enligt nuvarande design (20141111)
	\end{itemize}
	
	\item Skriva klart rapporten inför måndag
	
	\end{itemize}
	% MAIN END
	
	

	
    
    \subsection*{Vecka 3}
    \begin{itemize}
    \item Lämna in första rapporten (måndag).
	\item Skriv kravspecifikation för regulatorn och mjukvaran.
    \item Design av regulatorn
    \item Få sensorerna att fungera
    \begin{itemize}
    	\item Testa sensor gyroskop
	\item Testa sensor accelerometer
	\item Undersöka om vi behöver flera sensorer
    \end{itemize}
    \item Utföra vinkel skattning med gyroskop sensorn (enkelt experiment)
    \item Test av regulatorn via simulink
    \item Börja på implementringskoden. Allmänt test så att komponenterna fungerar samtidigt.
	\end{itemize}
    
    \subsection*{Vecka 4}
    \begin{itemize}
	\item Design av regulatorn.
    \item Test av designen.
    \item Högnivådesign för implementering av regulatorn.
    \item Börja på implementering av regulatorn.
	\end{itemize}
    
    \subsection*{Vecka 5}
    \begin{itemize}
    \item Skicka in andra rapporten (Fredag)
	\item Implementation av regulatorn.
    \item Slutlig design av regulatorn.
    \item Slutligt test av regulatorns design.
	\end{itemize}
    
    \subsection*{Vecka 6}
    \begin{itemize}
	\item Slutlig implementation av regulatorn.
    \item Förbättra regulatordesignen vid behov.
    \item Påbörja den slutliga rapporten.
	\end{itemize}
    
    \subsection*{Vecka 7}
    \begin{itemize}
	\item Test av processen.
    \item Testa hjulen oberoende.
    \item Skriv slutlig rapport.
	\end{itemize}

9 Januari: Skicka in slutrapport

\section{Arbetsområden}
	
    \subsection*{Kravspecifikation}

    \begin{itemize}
    \item [] Kraven för vad systemet ska klara av. Det innefattar fas marginal, tid för tillståndskonvergering osv.
    \end{itemize}
    
    \subsection*{Utveckling}
    
    \begin{itemize}
    \item [] Design av regulatorn
    \item [] Implemeteringen av regulator via C.
    \end{itemize}
    
    \subsection*{Testning}
    
    \begin{itemize}
    \item [] Test av regulatorn via simulink
    \item [] Test av koden via den riktiga processen.
    \end{itemize}
    
    
\section{Arbetsbörda}

\begin{itemize}
\item[] Modellering av processen: 1 vecka

\item[] Bygge av enhjulingen: 1 vecka

\item[] Design av regulatorn: 3 veckor

\item[] Implementering av regulatorn: 3 veckor
\end{itemize}

\end{document}